\documentclass[11pt,xcolor={dvipsnames}]{beamer}
\usepackage[english]{babel}
\usepackage[latin1]{inputenc}
\usepackage{listings}
\usepackage{verbatim}
\usepackage{fancyvrb}
\usepackage{moreverb}
\usepackage{color}
\DefineVerbatimEnvironment%
    {term}{Verbatim}
    {fontsize=\footnotesize, commandchars=\\\{\},
     fontfamily=courier, fontseries=b, frame=single, framerule=0.2mm,
    framesep=2mm, rulecolor=\color{MidnightBlue}, 
    formatcom=\color{MidnightBlue}
    }
\newcommand{\tc}{\textcolor}
\newcommand{\key}[1]{\tc{orange}{#1}}
\newcommand{\rk}{\enskip{\key{$\hookleftarrow$}}}
\newcommand{\vs}{\vspace{1em}}
% lstset for Haskell in Texashold'em
\newcommand{\return}{\enskip$\hookleftarrow$}

\lstdefinestyle{mykeywords}
{ morekeywords={subsequences, comparing},
  deletekeywords=[1]{True, False, Bool, String, Char, Int, Integer, Float, Double, LT, GT, EQ},
  keywords=[2]{True, False, LT, GT, EQ, Ace, King, Queen, Jack, Ten, Nine, Eight, Seven, Six, Five, Four, Three, Two, HighCard,OnePair,TwoPairs,ThreeOfAKind,FullHouse,Straight,Flush,StraightFlush,FourOfAKind,RoyalFlush, Hearts, Spades, Diamonds, Clubs},
  keywords=[3]{Bool, Int, String, Char, Integer, Float, Double, Vector, Flour, Milk, Egg, Batter, Pancake, Suit, Card, Value, Rank, Category, Hand, Ranking, Entry, Display, Score}}


\lstdefinestyle{haskell}{classoffset=0,
    language=Haskell,
	basicstyle=\color{black}\ttfamily\small,
    columns=fullflexible,
    keepspaces=true,
    tabsize=4,
	tab=$\to$,
	literate={==}{$=\joinrel=$}3 {.o.}{$\circ$}2 {->}{$\to$ }2 {=>}{$\Longrightarrow$}1 {lll}{$\lambda$}1 {[]}{[\thinspace]}3 {__}{\space}1 
    {♣}{\large{\color{black}{\monoclubs}}}1 {♦}{\large{\color{red}{\monodiamonds}}}1 {♥}{\large{\color{red}{\monohearts}}}1 {♠}{\large{\color{black}{\monospades}}}1 , 
	numberstyle=\tiny,
	extendedchars,
	breaklines,
	frame=single,
	showtabs=false,
	showspaces=false,
	showstringspaces=false,
	keywordstyle=\color{NavyBlue},
    deletekeywords=[1]{True, False, Bool, String, Char, Int, Integer, Float, Double, LT, GT, EQ},
    keywordstyle=[2]\color{Green},
    keywordstyle=[3]\color{blue},
    keywordstyle=[4]\color{magenta},
	identifierstyle=\color{black},
    morekeywords={subsequences, comparing},
    keywords=[2]{EQ, LT, GT, True, False, Ace, King, Queen, Jack, Ten, Nine, Eight, Seven, Six, Five, Four, Three, Two, HighCard,OnePair,TwoPairs,ThreeOfAKind,FullHouse,Straight,Flush,StraightFlush,FourOfAKind,RoyalFlush, Hearts, Spades, Diamonds, Clubs},
    keywords=[3]{Bool, Int, String, Char, Integer, Float, Double, Vector, Flour, Milk, Egg, Batter, Pancake, Suit, Card, Value, Rank, Category, Hand, Ranking, Entry, Display, Score},
    keywords=[4]{type,data,Show,Ord,Eq,Enum},
	stringstyle=\color{RoyalPurple},
	commentstyle=\color{darkgray},
	captionpos=b,
	aboveskip=\smallskipamount,
    belowskip=\smallskipamount}

\lstdefinestyle{hspec}{classoffset=0,
    language=Haskell,
	basicstyle=\color{black}\ttfamily\small,
    rulecolor=\color{Green},
    frame=leftline,
	columns=flexible,
	tabsize=4,
	tab=$\to$,
	literate={==}{$=\joinrel=$}3 {.o.}{$\circ$}2 {->}{$\to$ }2 {=>}{$\Longrightarrow$}1 {lll}{$\lambda$}1 {[]}{[ ]}2 {__}{\space}1 
    {♣}{\large{\color{black}{\monoclubs}}}1 {♦}{\large{\color{red}{\monodiamonds}}}1 {♥}{\large{\color{red}{\monohearts}}}1 {♠}{\large{\color{black}{\monospades}}}1 , 
	numberstyle=\tiny,
	extendedchars,
	breaklines,
	showtabs=false,
	showspaces=false,
	showstringspaces=false,
	keywordstyle=\color{NavyBlue},
    deletekeywords=[1]{True, False, Bool, String, Int, Integer, Float, Double, LT, GT, EQ},
    keywordstyle=[2]\color{Green},
    keywordstyle=[3]\color{blue},
	identifierstyle=\color{black},
    morekeywords={subsequences, comparing, shouldBe, it, describe},
    keywords=[2]{EQ, LT, GT, True, False, Ace, King, Queen, Jack, Ten, Nine, Eight, Seven, Six, Five, Four, Three, Two, HighCard,OnePair,TwoPairs,ThreeOfAKind,FullHouse,Straight,Flush,StraightFlush,FourOfAKind,RoyalFlush, Hearts, Spades, Diamonds, Clubs},
    keywords=[3]{Bool, Int, String, Char, Integer, Float, Double, Vector, Flour, Milk, Egg, Batter, Pancake,Suit, Card, Value, Rank, Category, Hand, Ranking, Entry, Display, Score},
	stringstyle=\color{RoyalPurple},
	commentstyle=\color{darkgray},
	captionpos=b,
	aboveskip=\smallskipamount,
    belowskip=\smallskipamount}

\lstset{classoffset=0,
    language=Haskell,
	basicstyle=\color{black}\ttfamily\small,
    columns=fullflexible,
    keepspaces=true,
    tabsize=4,
	tab=$\to$,
    literate={==}{$=\joinrel=$}3 {.o.}{$\circ$}2 {->}{$\to$ }2 {=>}{$\Longrightarrow$}1 {lll}{$\lambda$}1 {[]}{[\thinspace]\thinspace}1 {__}{\space}1 
    {♣}{\large{\color{black}{\monoclubs}}}1 {♦}{\large{\color{red}{\monodiamonds}}}1 {♥}{\large{\color{red}{\monohearts}}}1 {♠}{\large{\color{black}{\monospades}}}1 , 
    style=mykeywords,    
	numberstyle=\tiny,
	float=tbph,
	extendedchars,
	breaklines,
	frame=none,
	showtabs=false,
	showspaces=false,
	showstringspaces=false,
	keywordstyle=\color{NavyBlue},
    keywordstyle=[2]\color{Green},
    keywordstyle=[3]\color{blue},
	identifierstyle=\color{black},
	stringstyle=\color{RoyalPurple},
	commentstyle=\color{darkgray},
	captionpos=b,
	aboveskip=\smallskipamount,
    belowskip=\smallskipamount}

\lstnewenvironment{haskell}[1][]
{\lstset{style=haskell,#1}}{}

\definecolor{teagreen}{rgb}{0.82, 0.94, 0.75}
\definecolor{seashell}{rgb}{1.0, 0.96, 0.93}
\definecolor{honeydew}{rgb}{0.94, 1.0, 0.94}
\lstnewenvironment{hspec}[1][]
{\lstset{style=hspec, #1}}{}
\definecolor{ghcicolor}{rgb}{0.98, 0.98, 0.98}
\lstnewenvironment{ghci}[1][]
{\lstset{classoffset=0,
    language=Haskell,
	basicstyle=\color{black}\ttfamily\small,
    backgroundcolor=\color{ghcicolor},
    rulecolor=\color{blue},
	columns=flexible,
	tabsize=4,
	tab=$\to$,
    literate={==}{$=\joinrel=$}3 {.A.}{{\color{gray} }}1 {.Q.}{{\color{gray}\return}}2 {.o.}{$\circ$}2 {->}{$\to$ }2 {=>}{$\Longrightarrow$}1 {lll}{$\lambda$}1 {[]}{[ ]}2 {__}{\space}1 
    {♣}{\large{\color{black}{\monoclubs}}}1 {♦}{\large{\color{red}{\monodiamonds}}}1 {♥}{\large{\color{red}{\monohearts}}}1 {♠}{\large{\color{black}{\monospades}}}1 , 
	numberstyle=\tiny,
	extendedchars,
	breaklines,
	frame=single,
	showtabs=false,
	showspaces=false,
	showstringspaces=false,
	keywordstyle=\color{NavyBlue},
    deletekeywords=[1]{True, False, Bool, String, Char, Int, Integer, Float, Double, LT, GT, EQ},
    keywordstyle=[2]\color{NavyBlue},
    keywordstyle=[3]\color{NavyBlue},
	identifierstyle=\color{NavyBlue},
    morekeywords={subsequences, comparing},
    keywords=[2]{C, H, EQ, LT, GT, True, False, Ace, King, Queen, Jack, Ten, Nine, Eight, Seven, Six, Five, Four, Three, Two, HighCard,OnePair,TwoPairs,ThreeOfAKind,FullHouse,Straight,Flush,StraightFlush,FourOfAKind,RoyalFlush, Hearts, Spades, Diamonds, Clubs},
    keywords=[3]{Bool, String, Char, Int, Integer, Float, Double, Vector, Flour, Milk, Egg, Batter, Pancake,Suit, Card, Value, Rank, Category, Hand, Ranking, Entry, Display, Score},
	stringstyle=\color{NavyBlue},
	commentstyle=\color{darkgray},
	captionpos=b,
	aboveskip=\smallskipamount,
    belowskip=\smallskipamount,
#1}}{}
\definecolor{lightgray}{rgb}{0.83, 0.83, 0.83}

\newcommand{\tsuccess}{\textcolor{Green}}
\newcommand{\terror}{\textcolor{orange}}
\newcommand{\tfailure}{\textcolor{red}}
\newcommand{\tblack}{\textcolor{black}}

\DefineVerbatimEnvironment%
    {term}{Verbatim}
    {frame=single, fontsize=\footnotesize, commandchars=\\\{\}}


\setbeamertemplate{footline}[frame number]
\begin{document}
\title{Slices of Functional Programming}
\subtitle{The Texas Hold'em Kata}
\author{Xavier Detant \& Christophe Thibaut}
\begin{frame}
\titlepage
\begin{center}
\texttt{git@github.com:ToF-/SlicesOfFP.git}\\
\texttt{SlicesOfFP/code/}
\end{center}
\end{frame}
\begin{frame}[fragile]
\frametitle{The Texas Hold'em Kata}
given this file: \texttt{input.txt} :
\begin{term}
    kc 9s ks kd 9d 3c 6d 
    9c ah ks kd 9d 3c 6d
    ac qc ks kd 9d 3c 
    9h 5s 
    4d 2d ks kd 9d 3c 6d
    7s ts ks kd 9d
\end{term}
after command\\ \texttt{runhaskell pokerhands.hs <input.txt}

then the output is
\begin{term}
    kc 9s ks kd 9d 3c 6d full house (winner)
    9c ah ks kd 9d 3c 6d two pair
    ac qc ks kd 9d 3c 
    9h 5s 
    4d 2d ks kd 9d 3c 6d flush
    7s ts ks kd 9d
\end{term}
\end{frame}
\begin{frame}[fragile]
\frametitle{The Texas Hold'em Kata}

in the line:\\
\begin{center}
\texttt{8s 9d Th Js Qd Kc Ah}
\end{center}
\texttt{T},\texttt{J},\texttt{Q},\texttt{K},\texttt{A} stand for \emph{Ten}, \emph{Jack}, \emph{Queen}, \emph{King}, \emph{Ace}, and\\
\texttt{h},\texttt{s},\texttt{d},\texttt{c} stand for \emph{Hearts}, \emph{Spades}, \emph{Diamonds}, \emph{Clubs}\\
\vs
Texas Hold'em in five steps:
\begin{enumerate}
\item Interpret Strings in terms of Cards
\item Compare Cards (by Rank or by Suit)
\item Find the Category of a Hand (Hand = group of 5 Cards)
\item Find the best possible Hand in a group of 7 Cards
\item Find the best player in a game
\end{enumerate}
\end{frame}
\begin{frame}[fragile]
\frametitle{program = function evaluation}
Lauch \emph{ghci} and try some functions:
\begin{term}
sqrt 1764\rk

Data.List.subsequences "ABCD"\rk

subtract 2 44\rk

 2 `subtract` 44\rk

subtract 1 (subtract 1 44)\rk

6 * (3 + 4)\rk

(*) 6 ((+) 3 4)\rk

Data.List.insert 42 [1,32,87]\rk
\end{term}
\end{frame}
\begin{frame}[fragile]
\frametitle{Writing a test}

A short program named \texttt{Specs.hs}:
\lstinputlisting[style=haskell,firstline=2]{../code/005.hs}
\vs
Running the test:
\begin{term}
runhaskell Specs.hs\rk
\end{term}
\end{frame}
\begin{frame}[fragile]
\frametitle{Writing a suite of tests}
Sequencing actions with ~do~:
\lstinputlisting[style=haskell,firstline=2]{../code/006.hs}
\begin{itemize}
\item the ~\$~ operator is an alternative to parentheses:
\item ~f \$ x y z~ $\equiv$ ~f (x y z)~
\item the ~do~ construct allows for sequencing of actions
\item the actions must be indented under their sequencing~do~ 
\item we will use ~do~ and actions only in the tests
\end{itemize}
\end{frame}
\begin{frame}[fragile]
\frametitle{Let's write some functions}
Write a function \emph{response} that passes this test:
\lstinputlisting[style=hspec,firstline=2]{../code/007.hs}
%stopzone
\end{frame}
\begin{frame}[fragile]
\frametitle{Pattern Matching}
\begin{haskell}
response 'Y' = True
response 'y' = True
response 'N' = False
response 'n' = False
\end{haskell}
\vs
Patterns allow for expressing distinct cases 
\end{frame}
\begin{frame}[fragile]
\frametitle{Pattern Matching}
Write a function ~label~ that passes this test:
\begin{hspec}
import Test.Hspec

main = hspec $ do
    describe "label" $ do
        it "should be an english label" $ do
            label "WO" `shouldBe` "Wool"
            label "CO" `shouldBe` "Cotton"
            label "PA" `shouldBe` "Nylon"
            label "PC" `shouldBe` "Acrylic"
            label "XX" `shouldBe` "--- unknown label ---"
            label "YY" `shouldBe` "--- unknown label ---"
import Test.Hspec
\end{hspec}
%stopzone
\end{frame}
\begin{frame}[fragile]
\frametitle{Pattern Matching}
\begin{haskell}
label "WO" = "Wool"
label "CO" = "Cotton"
label "PA" = "Nylon"
label "PC" = "Acrylic"
label   _  = "--- unknown label ---"
\end{haskell}
The underscore symbol in the left part of the equality denotes \emph{any value that is distinct from the values in the preceding patterns}.
\end{frame}
\begin{frame}[fragile]
\frametitle{Lists}
A way to collect values of the same type\\
Ghci:
\begin{term}
1 : 2 : 3 : []\rk

'a' : 'b' : 'c' : ""\rk

[4,8] ++ [0,7]\rk

head [4,8,0,7]\rk

tail [4,8,0,7]\rk

reverse "Hello World"\rk

concat ["A","List","Of","Lists"]\rk
\end{term}
\end{frame}
\begin{frame}[fragile]
\frametitle{Let's write some functions}
Write a function \emph{average} that passes this test:
\vs
\begin{hspec}
    describe "average" $ do
        it "should calculate the average" $ do
            average [ ]        `shouldBe` 0
            average [2, 4, 12] `shouldBe` 6
\end{hspec}
\end{frame}
\begin{frame}[fragile]
\frametitle{Let's write some functions}
using Pattern Matching to denote cases: 
\vs
\begin{haskell}
average [ ]  = 0
average xs   = sum xs `div` length xs
\end{haskell}
\vs
A variable defined in the left part of the equality receives the argument value and can be used in the right part.
\end{frame}
\begin{frame}[fragile]
\frametitle{Pattern Matching}
\begin{haskell}
ordered [a,b]   = a <= b
ordered [a,b,c] = ordered [a,b] && ordered [b,c]

product []     = 1
product (x:xs) = x * product xs
\end{haskell}
\vs
Patterns also allow for deconstructing data:
\begin{itemize}
\item elements of a list
\item head of a list and remaining list 
\end{itemize}
\end{frame}
\begin{frame}[fragile]
\frametitle{Comparing values}
Some useful checks about ~compare~:
\vs
\begin{hspec}
describe "compare" $ do
    it "should compare values of any type of class Ord" $ do
        compare 42 17       `shouldBe` GT
        compare 'A' 'B'     `shouldBe` LT
        compare 11.3 11.3   `shouldBe` EQ
        compare "cat" "dog" `shouldBe` LT
\end{hspec}
\end{frame}
\begin{frame}[fragile]
\frametitle{Strings are not Cards!}
There's no way that this test can pass:
\vs
\begin{hspec}
describe "using Strings as Cards" $ do
    it "cannot give satisfactory comparisons" $ do
        compare "Td" "Jc"  `shouldBe` LT
        compare "8d" "8c"  `shouldBe` EQ
        compare "Ah" "Jc"  `shouldBe` GT
\end{hspec}
\vs
unless we rewrite ~compare~ 
\end{frame}
\begin{frame}[fragile]
\frametitle{How to compare cards by rank ?}
Write a function ~rank~ that passes this test:
\vs
\begin{hspec}
describe "comparing card by rank" $ do
    it "should follow the rules of poker" $ do
        compare (rank "8d") (rank "6h") `shouldBe` GT
        compare (rank "4d") (rank "4h") `shouldBe` EQ
        compare (rank "9d") (rank "Th") `shouldBe` LT 
        compare (rank "Td") (rank "Jh") `shouldBe` LT 
        compare (rank "Jd") (rank "Qh") `shouldBe` LT 
        compare (rank "Qd") (rank "Kh") `shouldBe` LT 
        compare (rank "Kd") (rank "Ah") `shouldBe` LT 
\end{hspec}
Hint:
\begin{haskell}
rank ['A',_] = 14
rank ['K',_] = 13
    . . .
\end{haskell}
\end{frame}
\begin{frame}[fragile]
\frametitle{How to compare cards by suit}
Write a function ~suit~ that passes this test
\vs
\begin{hspec}
describe "comparing card by suit" $ do
    it "should follow the rules of poker" $ do
        suit "8d" == suit "6d" `shouldBe` True
        suit "4d" == suit "4h" `shouldBe` False
        suit "9d" == suit "Tc" `shouldBe` True
        suit "Td" == suit "Js" `shouldBe` False
\end{hspec}
\end{frame}
\begin{frame}[fragile]
\frametitle{Types}
Types are a way to check the meaning of programs\\
All expressions, all function definitions have a type.\\
Although Haskell can infer our types, we can explicitely declare function signatures:
\vs
\begin{haskell}
rank :: String -> Int
suit :: String -> Char
\end{haskell}
\end{frame}
\begin{frame}[fragile]
\frametitle{Types}
Thanks to types, expressions like
\begin{itemize}
\item  ~rank False~ 
\item  ~rank 3.1415~ 
\end{itemize}
are not legal
\vs
But:
\begin{itemize}
\item  ~rank "Foo"~ is still legal
\item  ~compare (rank "!*") (rank "18")~ == ... ?
\item  every ~String~ value is not a valid ~Card~ value
\item  only when comparing fails we know we had incorrect data
\end{itemize}
\end{frame}
\begin{frame}[fragile]
\frametitle{Tuples}
A way to gather values of different types
\vs
Ghci:
\begin{term}
:type (EQ,'@', False)\rk

:type ('A',True)\rk

:type fst\rk

:type snd\rk

fst ('A', True)\rk

snd ('A', True) \rk
\end{term}
\end{frame}
\begin{frame}[fragile]
\frametitle{a way to think about the problem}
Let's define types synonyms:
\begin{haskell}
type Card = (Rank, Suit) 
type Rank = Int
type Suit = Char

rank :: Card -> Rank
suit :: Card -> Suit
\end{haskell}
And a new function from ~String~ to ~Card~:
\begin{haskell}
card :: String -> Card
\end{haskell}
\end{frame}
\begin{frame}[fragile]
\frametitle{Comparing cards, improved}
Write the function: ~card :: String $\rightarrow$ Card~ 
so that the test pass
\begin{hspec}
describe "comparing card by rank" $ do
    it "should follow the rules of poker" $ do
        compare (rank (card "8d")) (rank (card "6h")) `shouldBe` GT
        compare (rank (card "4d")) (rank (card "4h")) `shouldBe` EQ
        compare (rank (card "9d")) (rank (card "Th")) `shouldBe` LT 
        compare (rank (card "Td")) (rank (card "Jh")) `shouldBe` LT 
        compare (rank (card "Jd")) (rank (card "Qh")) `shouldBe` LT 
        compare (rank (card "Qd")) (rank (card "Kh")) `shouldBe` LT 
        compare (rank (card "Kd")) (rank (card "Ah")) `shouldBe` LT 
\end{hspec}
Hint:
\begin{haskell}
card [r,s] = (charToRank r, charToSuit s)

charToRank 'A' = 14
charToRank 'K' = 13
...
\end{haskell}
\end{frame}
\begin{frame}[fragile]
\frametitle{Comparing cards, improved}
\begin{haskell}
card :: String -> Card
rank :: Card -> Rank
suit :: Card -> Suit
\end{haskell}
Better because:
\begin{itemize}
\item  once conversion is done, the comparing takes care of itself
\item  bad input is detected at conversion, not in comparisons
\end{itemize}
But:
\begin{itemize}
\item  you can still do silly things like ~rank (4807,'@')~ 
\end{itemize}
\end{frame}
\begin{frame}[fragile]
\frametitle{Type Class = a way to define type conformity}
Saying that
\begin{haskell}
data Rank = Two | Three | Four | Five | Six | Seven | Eight 
          | Nine | Ten | Jack | Queen | King | Ace
    deriving (Eq, Ord, Enum, Show)
\end{haskell}
means that values of type ~Rank~ 
\begin{itemize}
\item  can be compared with ~==~ and ~/=~ 
\item  can be compared with ~compare~, ~$<$~, ~$<$=~ ...
\item  can be converted to and fron Int with ~fromEnum~ and ~toEnum~
\item  can be converted to String with ~show~
\end{itemize}
\end{frame}
\begin{frame}[fragile]
\frametitle{Types = a way to think about a problem}
Let's create new types:
\begin{haskell}
data Suit = Hearts | Clubs | Diamonds | Spades
    deriving (Eq, Show)

data Rank = Two | Three | Four | Five | Six | Seven | Eight 
          | Nine | Ten | Jack | Queen | King | Ace
    deriving (Eq, Ord, Enum, Show)

type Card = (Rank, Suit)
\end{haskell}
Rewrite the ~card~ function so that the tests still pass
\vs
Hint:
\begin{haskell}
card [r,s] = (charToRank c, charToSuit s)

charToRank 'A' = toEnum 12
charToRank 'K' = toEnum 11
...
\end{haskell}
\end{frame}
\begin{frame}[fragile]
\frametitle{Type Class = a way to define type conformity}
Ghci:
\begin{term}
:load PokerHand.hs\rk

Two < Three\rk
Ace > King\rk
show "Queen"\rk
card "8d"\rk
\end{term}
Better design:
\begin{itemize}
\item  the type ~Card~ can have only 52 values.
\item  once conversion is done, you can only
\begin{itemize}
\item compare by rank order (no illegal rank allowed)
\item compare on equality by suit (no illegal suit allowed)
\end{itemize}
\end{itemize}
\end{frame}
\begin{frame}[fragile]
\frametitle{Checkpoint \#1}
\begin{center}
We have the proper types to describe our values\\
\vs
We have our first feature: comparing cards\\
\vs
\Large{Well Done!!}
\end{center}
\end{frame}
\begin{frame}[fragile]
\frametitle{Organizing Code in Modules}
Let's move the production code into its own \emph{module}:
\begin{haskell}[caption=PokerHand.hs]
module PokerHand
where
. . .
\end{haskell}
and use it in the Specs script:
\begin{hspec}
import Test.Hspec
import PokerHand
. . .
\end{hspec}
\end{frame}
\begin{frame}[fragile]
\frametitle{Passing Functions to Functions}
Ghci:
\begin{term}
import Data.Ord\rk

:type compare\rk

:type comparing\rk

comparing abs (-4) 3\rk

:load PokerHand.hs\rk

comparing rank (card "8c") (card "5d")\rk
\end{term}
the function ~rank~ is passed to the ~comparing~ function
\end{frame}
\begin{frame}[fragile]
\frametitle{Combining Functions}
Ghci:
\begin{term}
:type (.)\rk

(length . words) "time flies like an arrow"\rk

comparing (rank . card) "8c" "5d"\rk
\end{term}
~(f . g) x == f (g x)~ 
\end{frame}
\begin{frame}[fragile]
\frametitle{Combining Functions}
Refactor the test using ~comparing~ and the ~. ~ operator
\begin{hspec}[basicstyle=\color{black}\ttfamily\small]
describe "comparing card by rank" $ do
    it "should follow the rules of poker" $ do
        compare (rank (card "8d")) (rank (card "6h")) `shouldBe` GT
        compare (rank (card "4d")) (rank (card "4h")) `shouldBe` EQ
        compare (rank (card "9d")) (rank (card "Th")) `shouldBe` LT 
        compare (rank (card "Td")) (rank (card "Jh")) `shouldBe` LT 
        compare (rank (card "Jd")) (rank (card "Qh")) `shouldBe` LT 
        compare (rank (card "Qd")) (rank (card "Kh")) `shouldBe` LT 
        compare (rank (card "Kd")) (rank (card "Ah")) `shouldBe` LT 
\end{hspec}
\end{frame}
\begin{frame}[fragile]
\frametitle{Mapping a function to a list of values}
Ghci:
\begin{term}
:type map\rk
map negate [-34,42,17]\rk

map sqrt [1,2,3,4,5]\rk
\end{term}
\end{frame}
\begin{frame}[fragile]
\frametitle{Collecting Cards}
Write the function ~cards~ such that
\begin{hspec}
describe "cards" $ do
    it "should collect cards from a string" $ do
        cards "8d Ah Qc"  `shouldBe`
             [(Eight,Diamonds),(Ace,Hearts),(Queen,Clubs)]
\end{hspec}
\end{frame}
\begin{frame}[fragile]
\frametitle{Sorting}
\begin{term}
sort [42,3,17,1,22,4,38]\rk

sortBy compare "HELLO"\rk

sortBy (comparing length) (words "time flies like an arrow")\rk
\end{term}
\end{frame}
\begin{frame}[fragile]
\frametitle{Ranks of a hand}
Write the function `ranks` such that
\begin{hspec}
describe "ranks" $ do
    it "should give the sorted ranks of a hand" $ do
        ranks (cards "8d Ah Qc")  `shouldBe` [Ace, Queen, Eight]
\end{hspec}
\end{frame}
\begin{frame}[fragile]
\frametitle{Grouping}
\begin{term}
group "HELLO"\rk
    
(group . sort) "Cats and Dogs"\rk
\end{term}
\end{frame}
\begin{frame}[fragile]
\frametitle{Groups of Cards}
Write the function `groups` such that
\begin{hspec}
describe "groups" $ do
    it "should group and sort the ranks of a hand" $ do
        groups (cards "8d Ah Qc 8h 8s")  `shouldBe`
             [[Eight,Eight,Eight],[Ace],[Queen]]

        groups (cards "8d Ah Qc 8h As")  `shouldBe`
             [[Ace,Ace],[Eight,Eight],[Queen]]
\end{hspec}
Hint: use
\begin{itemize}
\item  ~sort~
\item  ~sortBy~
\item  ~comparing~
\item  ~group~
\item  ~reverse~
\end{itemize}
\end{frame}
\begin{frame}[fragile]
\frametitle{Categorizing groups of Cards}
A data type for Category
\begin{haskell}
data Category = HighCard | OnePair | TwoPairs | ThreeOfAKind 
              | Straight | Flush | FullHouse | FourOfAKind 
              | StraightFlush | RoyalFlush
    deriving (Eq,Ord,Show)
\end{haskell}
\end{frame}
\begin{frame}[fragile]
\frametitle{Categorizing groups of Cards}
Write the function
 ~category :: [[Rank]] -> Category~
\begin{hspec}
describe "category" $ do
    it "should determine the category of a hand" $ do
        let hs = ["4s 5d Kc Tc 3d"
                 ,"4s Kd Kc Tc 3d"
                 ,"4s Kd Kc Tc Td"
                 ,"Ts Kd Kc Kc 8d"
                 ,"Ts Kd Kc Tc Td"
                 ,"Ts Kd Kc Kc Kd"]
        map (category.groups.cards) hs ==  
                [HighCard, OnePair, TwoPairs
                ,ThreeOfAKind, FullHouse, FourOfAKind]
\end{hspec}
Hint:
\begin{haskell}
category [_,_,_,_,_]   = HighCard
category [[_,_],_,_,_] = OnePair
...
\end{haskell}
\end{frame}
\begin{frame}[fragile]
\frametitle{Special categories}
A ~Straight~ is like a ~HighCard~ with ranks forming a sequence\\
\vs
e.g. ~Th 9d 8c 7s 6s~
\vs
A ~Flush~ is like a ~HighCard~ with all cards of same suit
\vs
e.g. ~Kh Jh 9h 7h 6h~
\end{frame}
\begin{frame}[fragile]
\frametitle{Guards}
Pattern matching can be applied with conditions, called guards
\begin{haskell}
power n m | m >= 0    = product (replicate m n)
          | otherwise = error "negative exponent"   

sign n | n < 0 = -1
       | n > 0 =  1
       | _     =  0
\end{haskell}
\end{frame}
\begin{frame}[fragile]
\frametitle{Detecting a Flush}
Write the function ~isFlush~:
\begin{hspec}
describe "isFlush" $ do
    it "should detect when all cards have the same suit" $ do
        isFlush (cards "8d Ah 4d 3d Ad") `shouldBe` False
        isFlush (cards "8h Ah 4h 3h Kh") `shouldBe` True
\end{hspec}
Hint: use
\begin{itemize}
\item  ~group~
\item  ~length~
\item  pattern matching with guards
\end{itemize}
\end{frame}
\begin{frame}[fragile]
\frametitle{The Enum Type Class}
Ghci:
\begin{term}
fromEnum False\rk
fromEnum True\rk

:load PokerHand.hs\rk

fromEnum Ace\rk
fromEnum King\rk
\end{term}
\end{frame}
\begin{frame}[fragile]
\frametitle{Detecting a Straight}
Method:
\begin{itemize}
\item  Given a list of 5 distinct groups of 1 rank each, 
\item  And   the first rank value = the last rank value + 4
\item  Then  the category is Straight
\end{itemize}
\begin{haskell}
isStraight :: [Rank] -> Bool
isStraight [a,_,_,_,b] = fromEnum a == 4 + fromEnum b 
isStraight _               = False 
\end{haskell}
\end{frame}
\begin{frame}[fragile]
\frametitle{Lexicographic Order}
Tuples, like Lists can be compared according to lexicographic order:
\begin{center}
$(a,b) < (c,d) \equiv (a<c) \vee (a=c) \wedge (b<d)$\\
\vs
$[a,b] < [c,d] \equiv (a<c) \vee (a=c) \wedge (b<d)$
\end{center}
This allows for comparing hand by category then ranks:
\begin{itemize}
\item  If two hands have the same category, the winner is the hand with the highest rank in the category.
\item  If two hands have the same category and rank, the winner is the hand with the highest remaining cards. 
\end{itemize}
\end{frame}
\begin{frame}[fragile]
\frametitle{Comparing two hands}
Comparing two hands involves comparing their category, and if their categories are equal, comparing the ranks in the order given by the groups.\\
\vs
Creating values of type ~Ranking~ allows for such comparisons, provided that the ranks are sorted in reverse order.
\begin{haskell}
type Ranking = (Category, [Rank])

describe "Ranking" $ do
    it "should correctly compare two ranking values" $ do 
        (OnePair,[Ace,Ace,Ten,Eight,Five]) 
            > (OnePair, [Ace,Ace,Eight,Seven,Two])
             `shouldBe` True
\end{haskell}
%stopzone
\end{frame}
\begin{frame}[fragile]
\frametitle{Determining a Ranking}
Create the function:
\begin{haskell}[frame=none]
ranking :: [Cards] -> Ranking
\end{haskell}
\begin{hspec}
it "should keep the ranking of a hand" $ do
    ranking (cards "2c 2s 3s 3c 4h")
        `shouldBe` (TwoPairs, [Three,Three,Two,Two,Four])

    ranking (cards "2c 2s As 3c 4h")
        `shouldBe` (OnePair, [Two,Two,Ace,Four,Three])
\end{hspec}
%stopzone
Hint:
\begin{haskell}
ranking cs = (cat,rs)
where
    cat = category gs
    rs  = concat gs
    gs  =  ...
\end{haskell}
\end{frame}
\begin{frame}[fragile]
\frametitle{Special Categories (cont.)}
A \emph{Straight Flush} is a \emph{Straight} and a \emph{Flush} \\
\vs
e.g \texttt{Th 9h 8h 7h 6h}\\
\vs
A \emph{Royal Flush} is a \emph{Straight Flush} starting with an Ace\\
\vs
e.g. \texttt{Ah Kh Qh Jh Th}\\
\vs
\end{frame}
\begin{frame}[fragile]
\frametitle{Promoting to special categories}
\begin{haskell}
promote :: Ranking -> Ranking
promote (HighCard,[Ace,Five,_,_,_]) = (Straight,
                                      [Five,Four,Three,Two,Ace])
promote (HighCard,rs) | isStraight rs = (Straight, rs)
promote r = r

flushes :: Bool -> Ranking -> Ranking
flushes True (HighCard,rs) = (Flush, rs)
flushes True (Straight,[Ace,_,_,_,_]) = (RoyalFlush,
                                        [Ace,King,Queen,Jack,Ten])
flushes True (Straight,rs) = (StraightFlush, rs)
flushes False r = r
\end{haskell}
\end{frame}
\begin{frame}[fragile]
\frametitle{Ranking Final Test}
\begin{hspec}
it "should correctly order a list by ranking" $ do
    let s = ["7s 5c 4d 3d 2c" ,"As Kc Qd Jd 9c"
            ,"2h 2d 5c 4c 3c" ,"Ah Ad Kc Qc Jc"
            ,"2c 2s 3s 3c 4h" ,"Ac As Ks Kc Jh"
            ,"2h 2d 2c 4c 3c" ,"Ah Ad Ac Qc Jc"
            ,"5h 4s 3d 2c Ah" ,"Ah Ks Qd Jc Th"
            ,"7c 5c 4c 3c 2c" ,"Ac Kc Qc Jc 9c"
            ,"2h 2d 2c 3h 3c" ,"Ah Ad Ac Kh Kc"
            ,"2c 2s 2h 2d 3c" ,"Ac As Ah Ad Jc"
            ,"5c 4c 3c 2c Ac" ,"Ah Kh Qh Jh Th"]
        isOrdered [_] = True
        isOrdered (x:y:xs) = x < y && isOrdered (y:xs) 
        r = map (ranking.cards) s
    isOrdered r `shouldBe` True
\end{hspec}
%stopzone
\end{frame}
\begin{frame}[fragile]
\frametitle{Ranking Final Test}
Hint:
\begin{haskell}
ranking cs = flushes (isFlush cs) (promote (cat, rs))
where
cat = category gs
rs  = concat   gs
gs  = groups (ranks cs)
\end{haskell}
\end{frame}
\begin{frame}
\frametitle{Checkpoint \#2}
\begin{center}
We can compare two hands in Texas Hold'em\\
\vs
\Large{Well Done!!}
\end{center}
\end{frame}
\end{document}
