\documentclass[11pt,xcolor={dvipsnames}]{beamer}
\usepackage[english]{babel}
\usepackage[latin1]{inputenc}
\usepackage{listings}
\usepackage{verbatim}
\usepackage{fancyvrb}
\usepackage{moreverb}
\usepackage{color}
\DefineVerbatimEnvironment%
    {term}{Verbatim}
    {fontsize=\footnotesize, commandchars=\\\{\},
     fontfamily=courier, fontseries=b, frame=single, framerule=0.2mm,
    framesep=2mm, rulecolor=\color{MidnightBlue}, 
    formatcom=\color{MidnightBlue}
    }
\newcommand{\tc}{\textcolor}
\newcommand{\key}[1]{\tc{orange}{#1}}
\newcommand{\rk}{\enskip{\key{$\hookleftarrow$}}}
\newcommand{\vs}{\vspace{1em}}
\newcommand{\lstH}[1]{\lstinputlisting[style=haskell,firstline=2]{../code/#1.hs}}
\newcommand{\lstT}[1]{\lstinputlisting[style=hspec,firstline=2]{../code/#1.hs}}
% lstset for Haskell in Texashold'em
\newcommand{\return}{\enskip$\hookleftarrow$}

\lstdefinestyle{mykeywords}
{ morekeywords={subsequences, comparing},
  deletekeywords=[1]{True, False, Bool, String, Char, Int, Integer, Float, Double, LT, GT, EQ},
  keywords=[2]{True, False, LT, GT, EQ, Ace, King, Queen, Jack, Ten, Nine, Eight, Seven, Six, Five, Four, Three, Two, HighCard,OnePair,TwoPairs,ThreeOfAKind,FullHouse,Straight,Flush,StraightFlush,FourOfAKind,RoyalFlush, Hearts, Spades, Diamonds, Clubs},
  keywords=[3]{Bool, Int, String, Char, Integer, Float, Double, Vector, Flour, Milk, Egg, Batter, Pancake, Suit, Card, Value, Rank, Category, Hand, Ranking, Entry, Display, Score}}


\lstdefinestyle{haskell}{classoffset=0,
    language=Haskell,
	basicstyle=\color{black}\ttfamily\small,
    columns=fullflexible,
    keepspaces=true,
    tabsize=4,
	tab=$\to$,
	literate={==}{$=\joinrel=$}3 {.o.}{$\circ$}2 {->}{$\to$ }2 {=>}{$\Longrightarrow$}1 {lll}{$\lambda$}1 {[]}{[\thinspace]}3 {__}{\space}1 
    {♣}{\large{\color{black}{\monoclubs}}}1 {♦}{\large{\color{red}{\monodiamonds}}}1 {♥}{\large{\color{red}{\monohearts}}}1 {♠}{\large{\color{black}{\monospades}}}1 , 
	numberstyle=\tiny,
	extendedchars,
	breaklines,
	frame=single,
	showtabs=false,
	showspaces=false,
	showstringspaces=false,
	keywordstyle=\color{NavyBlue},
    deletekeywords=[1]{True, False, Bool, String, Char, Int, Integer, Float, Double, LT, GT, EQ},
    keywordstyle=[2]\color{Green},
    keywordstyle=[3]\color{blue},
    keywordstyle=[4]\color{magenta},
	identifierstyle=\color{black},
    morekeywords={subsequences, comparing},
    keywords=[2]{EQ, LT, GT, True, False, Ace, King, Queen, Jack, Ten, Nine, Eight, Seven, Six, Five, Four, Three, Two, HighCard,OnePair,TwoPairs,ThreeOfAKind,FullHouse,Straight,Flush,StraightFlush,FourOfAKind,RoyalFlush, Hearts, Spades, Diamonds, Clubs},
    keywords=[3]{Bool, Int, String, Char, Integer, Float, Double, Vector, Flour, Milk, Egg, Batter, Pancake, Suit, Card, Value, Rank, Category, Hand, Ranking, Entry, Display, Score},
    keywords=[4]{type,data,Show,Ord,Eq,Enum},
	stringstyle=\color{RoyalPurple},
	commentstyle=\color{darkgray},
	captionpos=b,
	aboveskip=\smallskipamount,
    belowskip=\smallskipamount}

\lstdefinestyle{hspec}{classoffset=0,
    language=Haskell,
	basicstyle=\color{black}\ttfamily\small,
    rulecolor=\color{Green},
    frame=leftline,
	columns=flexible,
	tabsize=4,
	tab=$\to$,
	literate={==}{$=\joinrel=$}3 {.o.}{$\circ$}2 {->}{$\to$ }2 {=>}{$\Longrightarrow$}1 {lll}{$\lambda$}1 {[]}{[ ]}2 {__}{\space}1 
    {♣}{\large{\color{black}{\monoclubs}}}1 {♦}{\large{\color{red}{\monodiamonds}}}1 {♥}{\large{\color{red}{\monohearts}}}1 {♠}{\large{\color{black}{\monospades}}}1 , 
	numberstyle=\tiny,
	extendedchars,
	breaklines,
	showtabs=false,
	showspaces=false,
	showstringspaces=false,
	keywordstyle=\color{NavyBlue},
    deletekeywords=[1]{True, False, Bool, String, Int, Integer, Float, Double, LT, GT, EQ},
    keywordstyle=[2]\color{Green},
    keywordstyle=[3]\color{blue},
	identifierstyle=\color{black},
    morekeywords={subsequences, comparing, shouldBe, it, describe},
    keywords=[2]{EQ, LT, GT, True, False, Ace, King, Queen, Jack, Ten, Nine, Eight, Seven, Six, Five, Four, Three, Two, HighCard,OnePair,TwoPairs,ThreeOfAKind,FullHouse,Straight,Flush,StraightFlush,FourOfAKind,RoyalFlush, Hearts, Spades, Diamonds, Clubs},
    keywords=[3]{Bool, Int, String, Char, Integer, Float, Double, Vector, Flour, Milk, Egg, Batter, Pancake,Suit, Card, Value, Rank, Category, Hand, Ranking, Entry, Display, Score},
	stringstyle=\color{RoyalPurple},
	commentstyle=\color{darkgray},
	captionpos=b,
	aboveskip=\smallskipamount,
    belowskip=\smallskipamount}

\lstset{classoffset=0,
    language=Haskell,
	basicstyle=\color{black}\ttfamily\small,
    columns=fullflexible,
    keepspaces=true,
    tabsize=4,
	tab=$\to$,
    literate={==}{$=\joinrel=$}3 {.o.}{$\circ$}2 {->}{$\to$ }2 {=>}{$\Longrightarrow$}1 {lll}{$\lambda$}1 {[]}{[\thinspace]\thinspace}1 {__}{\space}1 
    {♣}{\large{\color{black}{\monoclubs}}}1 {♦}{\large{\color{red}{\monodiamonds}}}1 {♥}{\large{\color{red}{\monohearts}}}1 {♠}{\large{\color{black}{\monospades}}}1 , 
    style=mykeywords,    
	numberstyle=\tiny,
	float=tbph,
	extendedchars,
	breaklines,
	frame=none,
	showtabs=false,
	showspaces=false,
	showstringspaces=false,
	keywordstyle=\color{NavyBlue},
    keywordstyle=[2]\color{Green},
    keywordstyle=[3]\color{blue},
	identifierstyle=\color{black},
	stringstyle=\color{RoyalPurple},
	commentstyle=\color{darkgray},
	captionpos=b,
	aboveskip=\smallskipamount,
    belowskip=\smallskipamount}

\lstnewenvironment{haskell}[1][]
{\lstset{style=haskell,#1}}{}

\definecolor{teagreen}{rgb}{0.82, 0.94, 0.75}
\definecolor{seashell}{rgb}{1.0, 0.96, 0.93}
\definecolor{honeydew}{rgb}{0.94, 1.0, 0.94}
\lstnewenvironment{hspec}[1][]
{\lstset{style=hspec, #1}}{}
\definecolor{ghcicolor}{rgb}{0.98, 0.98, 0.98}
\lstnewenvironment{ghci}[1][]
{\lstset{classoffset=0,
    language=Haskell,
	basicstyle=\color{black}\ttfamily\small,
    backgroundcolor=\color{ghcicolor},
    rulecolor=\color{blue},
	columns=flexible,
	tabsize=4,
	tab=$\to$,
    literate={==}{$=\joinrel=$}3 {.A.}{{\color{gray} }}1 {.Q.}{{\color{gray}\return}}2 {.o.}{$\circ$}2 {->}{$\to$ }2 {=>}{$\Longrightarrow$}1 {lll}{$\lambda$}1 {[]}{[ ]}2 {__}{\space}1 
    {♣}{\large{\color{black}{\monoclubs}}}1 {♦}{\large{\color{red}{\monodiamonds}}}1 {♥}{\large{\color{red}{\monohearts}}}1 {♠}{\large{\color{black}{\monospades}}}1 , 
	numberstyle=\tiny,
	extendedchars,
	breaklines,
	frame=single,
	showtabs=false,
	showspaces=false,
	showstringspaces=false,
	keywordstyle=\color{NavyBlue},
    deletekeywords=[1]{True, False, Bool, String, Char, Int, Integer, Float, Double, LT, GT, EQ},
    keywordstyle=[2]\color{NavyBlue},
    keywordstyle=[3]\color{NavyBlue},
	identifierstyle=\color{NavyBlue},
    morekeywords={subsequences, comparing},
    keywords=[2]{C, H, EQ, LT, GT, True, False, Ace, King, Queen, Jack, Ten, Nine, Eight, Seven, Six, Five, Four, Three, Two, HighCard,OnePair,TwoPairs,ThreeOfAKind,FullHouse,Straight,Flush,StraightFlush,FourOfAKind,RoyalFlush, Hearts, Spades, Diamonds, Clubs},
    keywords=[3]{Bool, String, Char, Int, Integer, Float, Double, Vector, Flour, Milk, Egg, Batter, Pancake,Suit, Card, Value, Rank, Category, Hand, Ranking, Entry, Display, Score},
	stringstyle=\color{NavyBlue},
	commentstyle=\color{darkgray},
	captionpos=b,
	aboveskip=\smallskipamount,
    belowskip=\smallskipamount,
#1}}{}
\definecolor{lightgray}{rgb}{0.83, 0.83, 0.83}

\newcommand{\tsuccess}{\textcolor{Green}}
\newcommand{\terror}{\textcolor{orange}}
\newcommand{\tfailure}{\textcolor{red}}
\newcommand{\tblack}{\textcolor{black}}

\DefineVerbatimEnvironment%
    {term}{Verbatim}
    {frame=single, fontsize=\footnotesize, commandchars=\\\{\}}


\setbeamertemplate{footline}[frame number]
\begin{document}
\title{Slices of Functional Programming}
\subtitle{The Texas Hold'em Kata}
\author{Xavier Detant \& Christophe Thibaut}
% SLIDE  001
\begin{frame}
\titlepage
\begin{center}
\texttt{git@github.com:ToF-/SlicesOfFP.git}\\
\texttt{SlicesOfFP/code/}
\end{center}
\end{frame}
% SLIDE  002
\begin{frame}[fragile]
\frametitle{The Texas Hold'em Kata}
given this file: \texttt{input.txt} :
\begin{term}
    kc 9s ks kd 9d 3c 6d 
    9c ah ks kd 9d 3c 6d
    ac qc ks kd 9d 3c 
    9h 5s 
    4d 2d ks kd 9d 3c 6d
    7s ts ks kd 9d
\end{term}
after command\\ \texttt{runhaskell pokerhands.hs <input.txt}

then the output is
\begin{term}
    kc 9s ks kd 9d 3c 6d full house (winner)
    9c ah ks kd 9d 3c 6d two pair
    ac qc ks kd 9d 3c 
    9h 5s 
    4d 2d ks kd 9d 3c 6d flush
    7s ts ks kd 9d
\end{term}
\end{frame}
% SLIDE  003
\begin{frame}[fragile]
\frametitle{The Texas Hold'em Kata}

in the line:\\
\begin{center}
\texttt{8s 9d Th Js Qd Kc Ah}
\end{center}
\texttt{T},\texttt{J},\texttt{Q},\texttt{K},\texttt{A} stand for \emph{Ten}, \emph{Jack}, \emph{Queen}, \emph{King}, \emph{Ace}, and\\
\texttt{h},\texttt{s},\texttt{d},\texttt{c} stand for \emph{Hearts}, \emph{Spades}, \emph{Diamonds}, \emph{Clubs}\\
\vs
Texas Hold'em in five steps:
\begin{enumerate}
\item Interpret Strings in terms of Cards
\item Compare Cards (by Rank or by Suit)
\item Find the Category of a Hand (Hand = group of 5 Cards)
\item Find the best possible Hand in a group of 7 Cards
\item Find the best player in a game
\end{enumerate}
\end{frame}
% SLIDE  004
\begin{frame}[fragile]
\frametitle{program = function evaluation}
Launch \emph{ghci} and try some functions:
\begin{term}
sqrt 1764\rk

Data.List.subsequences "ABCD"\rk

subtract 2 44\rk

 2 `subtract` 44\rk

subtract 1 (subtract 1 44)\rk

6 * (3 + 4)\rk

(*) 6 ((+) 3 4)\rk

Data.List.insert 42 [1,32,87]\rk
\end{term}
\end{frame}
% SLIDE  005
\begin{frame}[fragile]
\frametitle{Writing a test}

\begin{term}
cabal install hspec\rk
\end{term}
A short program named \texttt{Specs.hs}:
\lstT{005}
\vs
Running the test:
\begin{term}
runhaskell Specs.hs\rk
\end{term}
\end{frame}
% SLIDE  006
\begin{frame}[fragile]
\frametitle{Writing a suite of tests}
Sequencing actions with ~do~:
\lstT{006}
\begin{itemize}
\item the ~\$~ operator is an alternative to parentheses:
\item ~f \$ x y z~ $\equiv$ ~f (x y z)~
\item the ~do~ construct allows for sequencing of actions
\item the actions must be indented under their sequencing~do~ 
\item we will use ~do~ and actions only in the tests
\end{itemize}
\end{frame}
% SLIDE  007
\begin{frame}[fragile]
\frametitle{Organizing Code in Modules}
Let's create a module ~PokerHand~ in a file named \emph{PokerHand.hs}:
\lstH{029a}
and use it in the Specs script:
\lstT{029b}
Write a function \emph{response} that passes this test:
\lstT{007}
%stopzone
\end{frame}
% SLIDE  008
\begin{frame}[fragile]
\frametitle{Pattern Matching}
\lstH{008}
\vs
Patterns allow for expressing distinct cases 
\end{frame}
% SLIDE  009
\begin{frame}[fragile]
\frametitle{Pattern Matching}
Write a function ~label~ that passes this test:

\lstT{009}
%stopzone
\end{frame}
% SLIDE  010
\begin{frame}[fragile]
\frametitle{Pattern Matching}
\begin{haskell}
label "WO" = "Wool"
label "CO" = "Cotton"
label "PA" = "Nylon"
label "PC" = "Acrylic"
label   _  = "--- unknown label ---"
\end{haskell}
The underscore symbol in the left part of the equality denotes \emph{any value that is distinct from the values in the preceding patterns}.
\end{frame}
% SLIDE  011
\begin{frame}[fragile]
\frametitle{Lists}
A way to collect values of the same type\\
Ghci:
\begin{term}
1 : 2 : 3 : []\rk

'a' : 'b' : 'c' : ""\rk

[4,8] ++ [0,7]\rk

head [4,8,0,7]\rk

tail [4,8,0,7]\rk

reverse "Hello World"\rk

concat ["A","List","Of","Lists"]\rk
\end{term}
\end{frame}
% SLIDE  012
\begin{frame}[fragile]
\frametitle{Let's write some functions}
Write a function \emph{average} that passes this test:
\vs
\lstT{012}
\end{frame}
% SLIDE  013
\begin{frame}[fragile]
\frametitle{Let's write some functions}
using Pattern Matching to denote cases: 
\vs
\lstH{013}
\vs
A variable defined in the left part of the equality receives the argument value and can be used in the right part.
\end{frame}
% SLIDE  014
\begin{frame}[fragile]
\frametitle{Pattern Matching}
\lstH{014}
\vs
Patterns also allow for deconstructing data:
\begin{itemize}
\item elements of a list
\item head of a list and remaining list 
\end{itemize}
\end{frame}
% SLIDE  015
\begin{frame}[fragile]
\frametitle{Comparing values}
Some useful checks about ~compare~:
\vs
\lstT{015}
\end{frame}
% SLIDE  016
\begin{frame}[fragile]
\frametitle{Strings are not Cards!}
There's no way that this test can pass:
\vs
\lstT{016}
\vs
unless we rewrite ~compare~ 
\end{frame}
% SLIDE  017
\begin{frame}[fragile]
\frametitle{How to compare cards by rank ?}
Write a function ~rank~ that passes this test:
\vs
\lstT{017}
Hint:
\begin{haskell}
rank ['A',_] = 14
rank ['K',_] = 13
    . . .
\end{haskell}
\end{frame}
% SLIDE  018
\begin{frame}[fragile]
\frametitle{How to compare cards by suit}
Write a function ~suit~ that passes this test
\vs
\lstT{018}
\end{frame}
% SLIDE  019
\begin{frame}[fragile]
\frametitle{Types}
Types are a way to check the meaning of programs\\
All expressions, all function definitions have a type.\\
Although Haskell can infer our types, we can explicitely declare function signatures:
\vs
\begin{haskell}
rank :: String -> Int
suit :: String -> Char
\end{haskell}
\end{frame}
% SLIDE  020
\begin{frame}[fragile]
\frametitle{Types}
Thanks to types, expressions like
\begin{itemize}
\item  ~rank False~ 
\item  ~rank 3.1415~ 
\end{itemize}
are not legal
\vs
But:
\begin{itemize}
\item  ~rank "Foo"~ is still legal
\item  ~compare (rank "!*") (rank "18")~ == ... ?
\item  every ~String~ value is not a valid ~Card~ value
\item  only when comparing fails we know we had incorrect data
\end{itemize}
\end{frame}
% SLIDE  021
\begin{frame}[fragile]
\frametitle{Tuples}
A way to gather values of different types
\vs
Ghci:
\begin{term}
:type (EQ,'@', False)\rk

:type ('A',True)\rk

:type fst\rk

:type snd\rk

fst ('A', True)\rk

snd ('A', True) \rk
\end{term}
\end{frame}
% SLIDE  022
\begin{frame}[fragile]
\frametitle{a way to think about the problem}
Let's define types synonyms:
\lstH{022}
And a new function from ~String~ to ~Card~:
\begin{haskell}
card :: String -> Card
\end{haskell}
\end{frame}
% SLIDE  023
\begin{frame}[fragile]
\frametitle{Comparing cards, improved}
Write the function: ~card :: String $\rightarrow$ Card~ 
so that the test pass
\lstT{023}
Hint:
\begin{haskell}
card [r,s] = (charToRank r, charToSuit s)

charToRank 'A' = 14
charToRank 'K' = 13
...
\end{haskell}
\end{frame}
% SLIDE  024
\begin{frame}[fragile]
\frametitle{Comparing cards, improved}
\lstH{024}
Better because:
\begin{itemize}
\item  once conversion is done, the comparing takes care of itself
\item  bad input is detected at conversion, not in comparisons
\end{itemize}
But:
\begin{itemize}
\item  you can still do silly things like ~rank (4807,'@')~ 
\end{itemize}
\end{frame}
% SLIDE  025
\begin{frame}[fragile]
\frametitle{Type Class = a way to define type conformity}
Here's a new type:
\begin{haskell}
data Rank = Two | Three | Four | Five | Six | Seven | Eight 
          | Nine | Ten | Jack | Queen | King | Ace
    deriving (Eq, Ord, Enum, Show)
\end{haskell}
Using ~deriving (Eq, Ord, Enum,  Show)~ means that values of type ~Rank~ 
\begin{itemize}
\item  can be compared with ~==~ and ~/=~ 
\item  can be compared with ~compare~, ~$<$~, ~$<$=~ ...
\item  can be converted to and fron Int with ~fromEnum~ and ~toEnum~
\item  can be converted to String with ~show~
\end{itemize}
\end{frame}
% SLIDE  026
\begin{frame}[fragile]
\frametitle{Types = a way to think about a problem}
Let's create new types:
\begin{haskell}
data Suit = Hearts | Clubs | Diamonds | Spades
    deriving (Eq, Show)

data Rank = Two | Three | Four | Five | Six | Seven | Eight 
          | Nine | Ten | Jack | Queen | King | Ace
    deriving (Eq, Ord, Enum, Show)

type Card = (Rank, Suit)
\end{haskell}
Rewrite the ~card~ function so that the tests still pass
\vs
Hint:
\begin{haskell}
card [r,s] = (charToRank c, charToSuit s)

charToRank 'A' = toEnum 12
charToRank 'K' = toEnum 11
...
\end{haskell}
\end{frame}
% SLIDE  027
\begin{frame}[fragile]
\frametitle{Type Class = a way to define type conformity}
Ghci:
\begin{term}
:load PokerHand.hs\rk

Two < Three\rk
Ace > King\rk
show Queen\rk
card "8d"\rk
\end{term}
Better design:
\begin{itemize}
\item  the type ~Card~ can have only 52 values.
\item  once conversion is done, you can only
\begin{itemize}
\item compare by rank order (no illegal rank allowed)
\item compare on equality by suit (no illegal suit allowed)
\end{itemize}
\end{itemize}
\end{frame}
% SLIDE  028
\begin{frame}[fragile]
\frametitle{Checkpoint \#1}
\begin{center}
We have the proper types to describe our values\\
\vs
We have our first feature: comparing cards\\
\vs
\Large{Well Done!!}
\end{center}
\end{frame}
% SLIDE  029
\begin{frame}[fragile]
\frametitle{}
(this slide intentionally left blank)
\end{frame}
% SLIDE  030
\begin{frame}[fragile]
\frametitle{Passing Functions to Functions}
Ghci:
\begin{term}
import Data.Ord\rk

:type compare\rk

:type comparing\rk

comparing abs (-4) 3\rk

:load PokerHand.hs\rk

comparing rank (card "8c") (card "5d")\rk
\end{term}
the function ~rank~ is passed to the ~comparing~ function
\end{frame}
% SLIDE  031
\begin{frame}[fragile]
\frametitle{Combining Functions}
Ghci:
\begin{term}
:type (.)\rk

(length . words) "time flies like an arrow"\rk

comparing (rank . card) "8c" "5d"\rk
\end{term}
~(f . g) x == f (g x)~ 
\end{frame}
% SLIDE  032
\begin{frame}[fragile]
\frametitle{Combining Functions}
Refactor the test using ~comparing~ and the ~. ~ operator
\lstT{032}
\end{frame}
% SLIDE  033
\begin{frame}[fragile]
\frametitle{Mapping a function to a list of values}
Ghci:
\begin{term}
:type map\rk
map negate [-34,42,17]\rk

map sqrt [1,2,3,4,5]\rk
\end{term}
\end{frame}
% SLIDE  034
\begin{frame}[fragile]
\frametitle{Collecting Cards}
Write the function ~cards~ such that
\lstT{034}
\end{frame}
% SLIDE  035
\begin{frame}[fragile]
\frametitle{Sorting}
\begin{term}
import Data.List\rk

sort [42,3,17,1,22,4,38]\rk

sortBy compare "HELLO"\rk

sortBy (comparing length) (words "time flies like an arrow")\rk

:type flip\rk

flip compare 4 5\rk

sortBy (flip compare) "HELLO"\rk
\end{term}
\end{frame}
% SLIDE  036
\begin{frame}[fragile]
\frametitle{Ranks of a hand}
Write the function `ranks` such that
\lstT{036}
\end{frame}
% SLIDE  037
\begin{frame}[fragile]
\frametitle{Grouping}
\begin{term}
group "HELLO"\rk
    
(group . sort) "Cats and Dogs"\rk
\end{term}
\end{frame}
% SLIDE  038
\begin{frame}[fragile]
\frametitle{Groups of Cards}
Write the function `groups` such that
\lstT{038}
Hint: use
\begin{itemize}
\item  ~sort~
\item  ~sortBy~
\item  ~comparing~
\item  ~group~
\item  ~reverse~
\end{itemize}
\end{frame}
% SLIDE  039
\begin{frame}[fragile]
\frametitle{Categorizing groups of Cards}
A data type for Category
\lstH{039}
\end{frame}
% SLIDE  040
\begin{frame}[fragile]
\frametitle{Categorizing groups of Cards}
Write the function
 ~category :: [[Rank]] $\rightarrow$ Category~
\lstT{040}
Hint:
\begin{haskell}
category [_,_,_,_,_]   = HighCard
category [[_,_],_,_,_] = OnePair
...
\end{haskell}
\end{frame}
% SLIDE  041
\begin{frame}[fragile]
\frametitle{Special categories}
A ~Straight~ is like a ~HighCard~ with ranks forming a sequence\\
\vs
e.g. ~Th 9d 8c 7s 6s~\\
\vs
A ~Flush~ is like a ~HighCard~ with all cards of same suit\\
\vs
e.g. ~Kh Jh 9h 7h 6h~
\end{frame}
% SLIDE  042
\begin{frame}[fragile]
\frametitle{Guards}
Pattern matching can be applied with conditions, called guards
\begin{haskell}
power n m | m >= 0    = product (replicate m n)
          | otherwise = error "negative exponent"   

sign n | n < 0 = -1
       | n > 0 =  1
       | _     =  0
\end{haskell}
\end{frame}
% SLIDE  043
\begin{frame}[fragile]
\frametitle{Detecting a Flush}
Write the function ~isFlush~:
\lstT{043}
Hint: use
\begin{itemize}
\item  ~group~
\item  ~length~
\item  pattern matching with guards
\end{itemize}
\end{frame}
% SLIDE  044
\begin{frame}[fragile]
\frametitle{The Enum Type Class}
Ghci:
\begin{term}
fromEnum False\rk
fromEnum True\rk

:load PokerHand.hs\rk

fromEnum Ace\rk
fromEnum King\rk
\end{term}
\end{frame}
% SLIDE  045
\begin{frame}[fragile]
\frametitle{Detecting a Straight}
Method:
\begin{itemize}
\item  Given a list of 5 distinct groups of 1 rank each, 
\item  And   the first rank value = the last rank value + 4
\item  Then  the category is Straight
\end{itemize}
\lstH{045}
\end{frame}
% SLIDE  046
\begin{frame}[fragile]
\frametitle{Lexicographic Order}
Tuples, like Lists can be compared according to lexicographic order:
\begin{center}
$(a,b) < (c,d) \equiv (a<c) \vee (a=c) \wedge (b<d)$\\
\vs
$[a,b] < [c,d] \equiv (a<c) \vee (a=c) \wedge (b<d)$
\end{center}
This allows for comparing hand by category then ranks:
\begin{itemize}
\item  If two hands have the same category, the winner is the hand with the highest rank in the category.
\item  If two hands have the same category and rank, the winner is the hand with the highest remaining cards. 
\end{itemize}
\end{frame}
% SLIDE  047
\begin{frame}[fragile]
\frametitle{Comparing two hands}
Comparing two hands involves comparing their category, and if their categories are equal, comparing the ranks in the order given by the groups.\\
\vs
Creating values of type ~Ranking~ allows for such comparisons, provided that the ranks are sorted in reverse order.
\lstT{047}
%stopzone
\end{frame}
% SLIDE  048
\begin{frame}[fragile]
\frametitle{Determining a Ranking}
Create the function:
\begin{haskell}[frame=none]
ranking :: [Card] -> Ranking
\end{haskell}
\lstT{048}
%stopzone
Hint:
\begin{haskell}
ranking cs = (cat,rs)
where
    cat = category gs
    rs  = concat gs
    gs  =  ...
\end{haskell}
\end{frame}
% SLIDE  049
\begin{frame}[fragile]
\frametitle{Special Categories (cont.)}
A \emph{Straight Flush} is a \emph{Straight} and a \emph{Flush} \\
\vs
e.g \texttt{Th 9h 8h 7h 6h}\\
\vs
A \emph{Royal Flush} is a \emph{Straight Flush} starting with an Ace\\
\vs
e.g. \texttt{Ah Kh Qh Jh Th}\\
\vs
\end{frame}
% SLIDE  050
\begin{frame}[fragile]
\frametitle{Promoting to special categories}
\lstH{050}
\end{frame}
% SLIDE  051
\begin{frame}[fragile]
\frametitle{Ranking Final Test}
\begin{hspec}
it "should correctly order a list by ranking" $ do
    let s = ["7s 5c 4d 3d 2c" ,"As Kc Qd Jd 9c"
            ,"2h 2d 5c 4c 3c" ,"Ah Ad Kc Qc Jc"
            ,"2c 2s 3s 3c 4h" ,"Ac As Ks Kc Jh"
            ,"2h 2d 2c 4c 3c" ,"Ah Ad Ac Qc Jc"
            ,"5h 4s 3d 2c Ah" ,"Ah Ks Qd Jc Th"
            ,"7c 5c 4c 3c 2c" ,"Ac Kc Qc Jc 9c"
            ,"2h 2d 2c 3h 3c" ,"Ah Ad Ac Kh Kc"
            ,"2c 2s 2h 2d 3c" ,"Ac As Ah Ad Jc"
            ,"5c 4c 3c 2c Ac" ,"Ah Kh Qh Jh Th"]
        isOrdered [_] = True
        isOrdered (x:y:xs) = x < y && isOrdered (y:xs) 
        r = map (ranking.cards) s
    isOrdered r `shouldBe` True
\end{hspec}
%stopzone
\end{frame}
% SLIDE  052
\begin{frame}[fragile]
\frametitle{Ranking Final Test}
Hint:
\begin{haskell}
ranking cs = flushes (isFlush cs) (promote (cat, rs))
where
cat = category gs
rs  = concat   gs
gs  = groups   cs
\end{haskell}
\end{frame}
% SLIDE  053
\begin{frame}
\frametitle{Checkpoint \#2}
\begin{center}
We can compare two hands in Texas Hold'em\\
\vs
\Large{Well Done!!}
\end{center}
\end{frame}
\end{document}
